\documentclass{article}
\usepackage[T1,T2A]{fontenc}
\usepackage[utf8]{inputenc}
\usepackage[english,ukrainian]{babel}
\usepackage[]{amsthm} %lets us use \begin{proof}
\usepackage[]{amssymb} %gives us the character \varnothing

\begin{document}

\title{Домашка 3. Відношення еквівалентності, конгруенція}
\date{24 травня 2023}

\maketitle

\subsection*{Задача 1}
Чи є наступні відношення - відношеннями еквівалентності? Доведіть. Для відношень еквівалентності опишіть класи еквівалентності
\begin{itemize}
    \item $X=\{-1,0,1\} A=\mathcal{P}(X); u,v \in A; uRv$, якщо сума чисел в u дорівнює сумі чисел в v
    \item $s,t$ - строки довжини 4 з a і b; $sRt$, якщо перші два символи s і t однакові.
    \item $x,y \in \mathbb{R}; R = \{(x,y)|x-y \in \mathbb{N} \}$
    \item $A = \mathbb{R} \times \mathbb{R}$, $R$ відношення на $A$: $(x_1,y_1)R(x_2,y_2)$, якщо $x_1 = x_2$
    \item $L$ - множина прямих на площині $l_1,l_2 \in L; l_1 R l_2$, якщо $l_1=l_2$ або $l_1$ паралельно $l_2$
    \item  $n,m \in \mathbb{N}; R = \{(n,m)|(n^2-m^2) \vdots 3\}$
\end{itemize}

\subsection*{Задача 2}
Наведіть приклад відношення еквівалентності на $\mathbb{N}$ у якого
\begin{itemize}
    \item 1 клас еквівалентності
    \item n класів еквівалентності
    \item нескінченна кількість класів еквівалентності
\end{itemize}

\subsection*{Задача 3}
$a\ mod\ n$ - невід'ємний залишок a від ділення на n. Обчисліть
\begin{itemize}
    \item $(7+7)\ mod\ 13$
    \item $(7*7)\ mod\ 13$
    \item $7^{12}\ mod\ 13$
    \item $7^{39}\ mod\ 13$
    \item $16^{51}\ mod\ 17$
    \item $16^{44}\ mod\ 17$
    \item $13^{21}\ mod\ 31$
\end{itemize}



\end{document}