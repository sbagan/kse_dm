\documentclass{article}
\usepackage[T1,T2A]{fontenc}
\usepackage[utf8]{inputenc}
\usepackage[english,ukrainian]{babel}
\usepackage[]{amsthm} %lets us use \begin{proof}
\usepackage[]{amssymb} %gives us the character \varnothing

\begin{document}

\title{Семінар 4. Бінарні відношення}
\date{12 травня 2023}

\maketitle

\subsection*{Задача 1}
Випишіть властивості бінарних відношень з лекції

\subsection*{Задача 2}
Для наступних відношень, вкажіть які властивості з задачі 1 виконуються, а які ні
\begin{itemize}
    \item Паралельність (на множині прямих в 2d)
    \item Симетрія відносно прямої (на множині точок в 2d)
    \item Рівність (на множині дійсних чисел)
    \item Взаємно простота (на множині натуральних чисел)
    \item Подільність ($a \vdots b$, на множині натуральних чисел)
    \item Підмножина ($a \subset b$)
    \item Не перетинаючись множини
    \item Незалежні події (на множині ймовірних подій)
\end{itemize}
Які відношення ви ще знаєте? Чи виконуються для них властивості з задачі 1

\subsection*{Задача 3}
Існує відома приказка, "ворог мого ворога - мій друг", придумайте відношення "дружності" і "ворожості", які б властивості для них виконувались?

\end{document}