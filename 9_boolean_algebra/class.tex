\documentclass{article}
\usepackage[T1,T2A]{fontenc}
\usepackage[utf8]{inputenc}
\usepackage[english,ukrainian]{babel}
\usepackage[]{amsthm} %lets us use \begin{proof}
\usepackage[]{amssymb} %gives us the character \varnothing

\begin{document}

\title{Семінар 9. Булеві функції}
\date{30 травня 2023}

\maketitle

\subsection*{Згадаємо лекцію}
\begin{itemize}
    \item Скільки існує одновимірних булевих функцій з 1 змінною? з 2? Які вони?
    \item Що таке ДДНФ? ДКНФ? Скільки існує різних ДДНФ?
\end{itemize}

\subsection*{ДДНФ}
Для наступних булевих функцій побудуйте таблицю істинності, за допомогою неї складіть ДДНФ
\begin{itemize}
    \item $x \oplus y$
    \item $x \lor (\lnot y \land z)$
    \item $(x~y) \land (x~z)$ (або $x=y=z$)
    \item $\lnot maj(x,y,z)$
    \item $\min(x,y,z)$
    \item $\max(x,y,z)$
\end{itemize}

\subsection*{Будування ДКНФ з ДДНФ}
Нехай $f(x,y,z) = x \oplus y \oplus z$
\begin{itemize}
    \item Складіть таблицю істинності для $f, \lnot f$
    \item Складіть ДДНФ для $\lnot f$
    \item Запишіть заперечення для ДДНФ
    \item Складіть ДКНФ
    \item Як можна було б побудувати ДКНФ з таблиці істинності?
\end{itemize}

\subsection*{ДКНФ}
Для наступних булевих функцій складіть ДКНФ
\begin{itemize}
    \item $x NOR y$ (або $\lnot (x \lor y)$) (або стрілка Пірса)
    \item $x \land (\lnot y \lor z)$
    \item $\lnot((x~y) \land (x~z))$ (або $x \neq y \neq z$)
    \item $maj(x,y,z)$
    \item $\max(x,y,z)$
    \item $\min(x,y,z)$
\end{itemize}

\end{document}