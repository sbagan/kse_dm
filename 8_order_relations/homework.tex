\documentclass{article}
\usepackage[T1,T2A]{fontenc}
\usepackage[utf8]{inputenc}
\usepackage[english,ukrainian]{babel}
\usepackage[]{amsthm} %lets us use \begin{proof}
\usepackage[]{amssymb} %gives us the character \varnothing

\begin{document}

\title{Домашка 4. Відношення порядку}
\date{30 травня 2023}

\maketitle

\subsection*{Задача 1}
Чи є наступні відношення частковими порядками, доведіть чи приведіть контрприклад до однієї з властивостей.
\begin{itemize}
    \item Нехай P множина всіх людей в світі, а відношення $aRb$ коли a є прямим предком b, або $a=b$.
    \item Нехай P множина всіх людей в світі, а відношення $aRb$ коли у a і b немає спільних друзів
    \item $n,m \in \mathbb{Z}$ $nRm$, якщо кожний простий дільник m є простим дільником n
    \item $n,m \in \mathbb{Z}; R = \{(n,m)|n+m -$ парне $\}$
\end{itemize}

\subsection*{Задача 2}
Доведіть, що наступні відношення - часткові порядки. Намалюйте діаграму Гассе для них. Вкажіть мінімальні, максимальні, найменьші, найбільші елементи (якщо немає вкажіть, що немає)
\begin{itemize}
    \item чисел $\{3, 5, 7, 11, 13, 16, 17\}$ і відношення кратно
    \item чисел $\{2, 3, 5, 10, 11, 15, 25\}$ і відношення кратно
    \item чисел $\{1, 3, 9, 27, 81, 243\}$ і відношення кратно
    \item $X=\{a,b,c\} A=\mathcal{P}(X)$ (булеан)$  R = \{(u,v) \in A| u \subseteq v\}$
    \item $X=\{a,b,c,d\} A=\mathcal{P}(X)$ (булеан)$  R = \{(u,v) \in A| u \subseteq v\}$
\end{itemize}

\subsection*{Задача 3}
Наведіть приклади відношень часткового порядку таких що:
\begin{itemize}
    \item є мінімальний елемент, але немає максимального
    \item є максимальний елемент, але немає мінімального
    \item немає ні максимального елемента, ні мінімального
\end{itemize}

\end{document}