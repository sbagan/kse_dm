\documentclass{article}
\usepackage[T1,T2A]{fontenc}
\usepackage[utf8]{inputenc}
\usepackage[english,ukrainian]{babel}
\usepackage[]{amsthm} %lets us use \begin{proof}
\usepackage[]{amssymb} %gives us the character \varnothing

\begin{document}

\title{Семінар 2. Індукція}
\date{28 квітня 2023}

\maketitle

\subsection*{Приклад}
Довести, що будь-яку сумму >=8 можна скласти з 3 і 5.
\subsection*{Означення}
Принцип математичної індукції
Нехай $P(n)$ це предикат визначений на всіх натуральних n, a - деяке натуральне число. Якщо насутпні два твердження вірні:
\begin{itemize}
    \item 1. $P(a)$ вірно (\textbf{база індукції})
    \item 2. Для всіх натуральних чисел $k > a$ з істинності $P(k)$ випливає істинність $P(k+1)$ (\textbf{індукційний перехід})
\end{itemize}
тоді для всіх натуральних $n >= a$ $P(n)$ вірно.
\subsection*{База важлива}
\begin{itemize}
    \item Що відбудеться, якщо замінити в монетках 8 на 7?
    \item Довести за допомогою математичної індукції, що для всіх натуральних чисел $3^n - 2$ парне
\end{itemize}
\subsection*{Задачі на послідовності}
\begin{itemize}
    \item Знайдіть $\sum (1 + 2 +... + n)$
    \item Знайдіть $\sum (1 + x + ... + x^n)$
    \item Знайдіть $\sum (1 + 4 +... + n^2)$ Підказка: ($n(n+1)(2n+1)/6$)
    \item Доведіть що $n^3 - n$ ділиться на 6 для всіх натуральних $n>2$
    \item Доведіть що $5^n + 9 < 6^n$ для всіх натуральних $n>1$
    \item Доведіть що $n^2 < 2^n$ для всіх натуральних $n > 4$
    \item Доведіть що $n! > n^2$ для всіх натуральних $n >= 4$
    \item Доведіть що $1 + nx <= (1 + x)^n$ для дійсних $x>-1$ і натуральних $n>=2$
\end{itemize}
\subsection*{Задача з зірочкою}
L-Триміно це фігурка з трьох квадратів у формі букви L. Доведіть за допомогою матиматичної індукції, що якщо з квадрата
розміром $2^n$ x $2^n$ прибрати одну будь-яку клітинку, то його можна повністю замостити за допомогою L-триміно без перекриття
\end{document}
