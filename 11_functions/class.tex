\documentclass{article}
\usepackage[T1,T2A]{fontenc}
\usepackage[utf8]{inputenc}
\usepackage[english,ukrainian]{babel}
\usepackage[]{amsthm} %lets us use \begin{proof}
\usepackage[]{amssymb} %gives us the character \varnothing
\usepackage{graphicx}

\begin{document}

\title{Семінар 11. Відображення}
\date{6 червня 2023}

\maketitle

\begin{figure}[ht!]
\centering
\includegraphics[width=90mm]{1}
\end{figure}

\begin{figure}
\centering
\includegraphics[width=90mm]{2}
\end{figure}

\begin{figure}
\centering
\includegraphics[width=90mm]{3}
\end{figure}

\begin{figure}
\centering
\includegraphics[width=90mm]{4}
\end{figure}

\begin{figure}
\centering
\includegraphics[width=90mm]{5}
\end{figure}

\begin{figure}
\centering
\includegraphics[width=90mm]{6}
\end{figure}

\begin{figure}
\centering
\includegraphics[width=90mm]{7}
\end{figure}

\begin{figure}
\centering
\includegraphics[width=90mm]{8}
\end{figure}

\begin{figure}
\centering
\includegraphics[width=90mm]{9}
\end{figure}

\begin{figure}
\centering
\includegraphics[width=90mm]{10}
\end{figure}

\begin{figure}
\centering
\includegraphics[width=90mm]{11}
\end{figure}

\begin{figure}
\centering
\includegraphics[width=90mm]{12}
\end{figure}

\pagebreak
\subsection*{Означення}
Дві множини $A,B$ (не обов`язково скінченні) називають рівнопотужними, якщо існує бієкція з однієї в іншу
\begin{itemize}
    \item Скінченні множини
    \item Натуральні числа і парні натуральні числа
    \item Натуральні числа і цілі числа
    \item Інтервал (0, 1) і інтервал (1, $+\inf$)
    \item Відрізок [0, 1] і відрізок [1, $+\inf$)
\end{itemize}

\end{document}